\documentclass[a4paper,12pt]{article}
\usepackage[text={18cm, 25cm}, centering, footskip=1pt]{geometry}
\usepackage[silent]{fontspec}
\usepackage{xltxtra}
\usepackage{xtab}
\usepackage[ancientgreek]{xgreek}
\setmainfont{Times New Roman}
\newfontfamily\greek[BoldFont={GFS Neohellenic Bold}, Scale=MatchLowercase
]{Times New Roman}
\newfontfamily\latin{Times New Roman}
\newfontfamily\Symbol{Cardo}
\pagestyle{empty}
\XeTeXinterchartokenstate=1
\newXeTeXintercharclass \symbols   
\XeTeXcharclass `^^^^0022 \symbols
\XeTeXcharclass `^^^^0023 \symbols
\XeTeXcharclass `^^^^003a \symbols
\XeTeXcharclass `^^^^003e \symbols
\XeTeXcharclass `^^^^005e \symbols
\XeTeXcharclass `^^^^007c \symbols
\XeTeXcharclass `^^^^007c \symbols
\XeTeXcharclass `^^^^00a2 \symbols
\XeTeXcharclass `^^^^00a9 \symbols
\XeTeXcharclass `^^^^00ac \symbols
\XeTeXcharclass `^^^^00b0 \symbols
\XeTeXcharclass `^^^^00c6 \symbols
\XeTeXcharclass `^^^^00e6 \symbols
\XeTeXcharclass `^^^^00f7 \symbols
\XeTeXcharclass `^^^^0110 \symbols
\XeTeXcharclass `^^^^0127 \symbols
\XeTeXcharclass `^^^^0130 \symbols
\XeTeXcharclass `^^^^0131 \symbols
\XeTeXcharclass `^^^^0152 \symbols
\XeTeXcharclass `^^^^0153 \symbols
\XeTeXcharclass `^^^^01a7 \symbols
\XeTeXcharclass `^^^^01b7 \symbols
\XeTeXcharclass `^^^^0222 \symbols
\XeTeXcharclass `^^^^0283 \symbols
\XeTeXcharclass `^^^^0292 \symbols
\XeTeXcharclass `^^^^02c6 \symbols
\XeTeXcharclass `^^^^02ca \symbols
\XeTeXcharclass `^^^^02cb \symbols
\XeTeXcharclass `^^^^02d9 \symbols
\XeTeXcharclass `^^^^0300 \symbols
\XeTeXcharclass `^^^^0301 \symbols
\XeTeXcharclass `^^^^0302 \symbols
\XeTeXcharclass `^^^^0305 \symbols
\XeTeXcharclass `^^^^0307 \symbols
\XeTeXcharclass `^^^^030a \symbols
\XeTeXcharclass `^^^^0313 \symbols
\XeTeXcharclass `^^^^0314 \symbols
\XeTeXcharclass `^^^^031a \symbols
\XeTeXcharclass `^^^^0323 \symbols
\XeTeXcharclass `^^^^0323 \symbols
\XeTeXcharclass `^^^^0328 \symbols
\XeTeXcharclass `^^^^032d \symbols
\XeTeXcharclass `^^^^033d \symbols
\XeTeXcharclass `^^^^0359 \symbols
\XeTeXcharclass `^^^^035c \symbols
\XeTeXcharclass `^^^^035c \symbols
\XeTeXcharclass `^^^^035d \symbols
\XeTeXcharclass `^^^^035e \symbols
\XeTeXcharclass `^^^^0361 \symbols
\XeTeXcharclass `^^^^0370 \symbols
\XeTeXcharclass `^^^^0375 \symbols
\XeTeXcharclass `^^^^03cf \symbols
\XeTeXcharclass `^^^^03fb \symbols
\XeTeXcharclass `^^^^03fd \symbols
\XeTeXcharclass `^^^^03fe \symbols
\XeTeXcharclass `^^^^03ff \symbols
\XeTeXcharclass `^^^^0485 \symbols
\XeTeXcharclass `^^^^0486 \symbols
\XeTeXcharclass `^^^^05d1 \symbols
\XeTeXcharclass `^^^^05e6 \symbols
\XeTeXcharclass `^^^^0660 \symbols
\XeTeXcharclass `^^^^0661 \symbols
\XeTeXcharclass `^^^^0662 \symbols
\XeTeXcharclass `^^^^0663 \symbols
\XeTeXcharclass `^^^^0664 \symbols
\XeTeXcharclass `^^^^0665 \symbols
\XeTeXcharclass `^^^^0666 \symbols
\XeTeXcharclass `^^^^0667 \symbols
\XeTeXcharclass `^^^^0668 \symbols
\XeTeXcharclass `^^^^0669 \symbols
\XeTeXcharclass `^^^^1dc0 \symbols
\XeTeXcharclass `^^^^1dc1 \symbols
\XeTeXcharclass `^^^^2013 \symbols
\XeTeXcharclass `^^^^2014 \symbols
\XeTeXcharclass `^^^^2016 \symbols
\XeTeXcharclass `^^^^2018 \symbols
\XeTeXcharclass `^^^^2026 \symbols
\XeTeXcharclass `^^^^2027 \symbols
\XeTeXcharclass `^^^^2027 \symbols
\XeTeXcharclass `^^^^2032 \symbols
\XeTeXcharclass `^^^^2035 \symbols
\XeTeXcharclass `^^^^2039 \symbols
\XeTeXcharclass `^^^^203a \symbols
\XeTeXcharclass `^^^^203b \symbols
\XeTeXcharclass `^^^^2042 \symbols
\XeTeXcharclass `^^^^204b \symbols
\XeTeXcharclass `^^^^2058 \symbols
\XeTeXcharclass `^^^^2059 \symbols
\XeTeXcharclass `^^^^205a \symbols
\XeTeXcharclass `^^^^205b \symbols
\XeTeXcharclass `^^^^205c \symbols
\XeTeXcharclass `^^^^205d \symbols
\XeTeXcharclass `^^^^205e \symbols
\XeTeXcharclass `^^^^208d \symbols
\XeTeXcharclass `^^^^208e \symbols
\XeTeXcharclass `^^^^20a4 \symbols
\XeTeXcharclass `^^^^210c \symbols
\XeTeXcharclass `^^^^2112 \symbols
\XeTeXcharclass `^^^^211e \symbols
\XeTeXcharclass `^^^^2126 \symbols
\XeTeXcharclass `^^^^2127 \symbols
\XeTeXcharclass `^^^^2135 \symbols
\XeTeXcharclass `^^^^2138 \symbols
\XeTeXcharclass `^^^^2183 \symbols
\XeTeXcharclass `^^^^2190 \symbols
\XeTeXcharclass `^^^^2190 \symbols
\XeTeXcharclass `^^^^2191 \symbols
\XeTeXcharclass `^^^^2192 \symbols
\XeTeXcharclass `^^^^2192 \symbols
\XeTeXcharclass `^^^^2208 \symbols
\XeTeXcharclass `^^^^2219 \symbols
\XeTeXcharclass `^^^^221a \symbols
\XeTeXcharclass `^^^^2227 \symbols
\XeTeXcharclass `^^^^2228 \symbols
\XeTeXcharclass `^^^^2234 \symbols
\XeTeXcharclass `^^^^2235 \symbols
\XeTeXcharclass `^^^^2237 \symbols
\XeTeXcharclass `^^^^223b \symbols
\XeTeXcharclass `^^^^223d \symbols
\XeTeXcharclass `^^^^2248 \symbols
\XeTeXcharclass `^^^^224c \symbols
\XeTeXcharclass `^^^^2263 \symbols
\XeTeXcharclass `^^^^2295 \symbols
\XeTeXcharclass `^^^^2297 \symbols
\XeTeXcharclass `^^^^22b8 \symbols
\XeTeXcharclass `^^^^22bb \symbols
\XeTeXcharclass `^^^^230a \symbols
\XeTeXcharclass `^^^^230a \symbols
\XeTeXcharclass `^^^^230b \symbols
\XeTeXcharclass `^^^^2310 \symbols
\XeTeXcharclass `^^^^2319 \symbols
\XeTeXcharclass `^^^^231c \symbols
\XeTeXcharclass `^^^^231d \symbols
\XeTeXcharclass `^^^^231e \symbols
\XeTeXcharclass `^^^^231f \symbols
\XeTeXcharclass `^^^^2329 \symbols
\XeTeXcharclass `^^^^232a \symbols
\XeTeXcharclass `^^^^239b \symbols
\XeTeXcharclass `^^^^239c \symbols
\XeTeXcharclass `^^^^239d \symbols
\XeTeXcharclass `^^^^239e \symbols
\XeTeXcharclass `^^^^239f \symbols
\XeTeXcharclass `^^^^23a0 \symbols
\XeTeXcharclass `^^^^23a7 \symbols
\XeTeXcharclass `^^^^23a8 \symbols
\XeTeXcharclass `^^^^23a9 \symbols
\XeTeXcharclass `^^^^23aa \symbols
\XeTeXcharclass `^^^^23aa \symbols
\XeTeXcharclass `^^^^23ab \symbols
\XeTeXcharclass `^^^^23ac \symbols
\XeTeXcharclass `^^^^23ad \symbols
\XeTeXcharclass `^^^^23d1 \symbols
\XeTeXcharclass `^^^^23d2 \symbols
\XeTeXcharclass `^^^^23d3 \symbols
\XeTeXcharclass `^^^^23d4 \symbols
\XeTeXcharclass `^^^^23d5 \symbols
\XeTeXcharclass `^^^^23d8 \symbols
\XeTeXcharclass `^^^^2573 \symbols
\XeTeXcharclass `^^^^25a1 \symbols
\XeTeXcharclass `^^^^25ad \symbols
\XeTeXcharclass `^^^^25ba \symbols
\XeTeXcharclass `^^^^25c4 \symbols
\XeTeXcharclass `^^^^25cb \symbols
\XeTeXcharclass `^^^^25cb \symbols
\XeTeXcharclass `^^^^25cc \symbols
\XeTeXcharclass `^^^^25cf \symbols
\XeTeXcharclass `^^^^2605 \symbols
\XeTeXcharclass `^^^^2609 \symbols
\XeTeXcharclass `^^^^260b \symbols
\XeTeXcharclass `^^^^260c \symbols
\XeTeXcharclass `^^^^260d \symbols
\XeTeXcharclass `^^^^2627 \symbols
\XeTeXcharclass `^^^^2629 \symbols
\XeTeXcharclass `^^^^263d \symbols
\XeTeXcharclass `^^^^263e \symbols
\XeTeXcharclass `^^^^263f \symbols
\XeTeXcharclass `^^^^2640 \symbols
\XeTeXcharclass `^^^^2642 \symbols
\XeTeXcharclass `^^^^2643 \symbols
\XeTeXcharclass `^^^^2644 \symbols
\XeTeXcharclass `^^^^2648 \symbols
\XeTeXcharclass `^^^^2649 \symbols
\XeTeXcharclass `^^^^264a \symbols
\XeTeXcharclass `^^^^264b \symbols
\XeTeXcharclass `^^^^264c \symbols
\XeTeXcharclass `^^^^264d \symbols
\XeTeXcharclass `^^^^264e \symbols
\XeTeXcharclass `^^^^264f \symbols
\XeTeXcharclass `^^^^2650 \symbols
\XeTeXcharclass `^^^^2651 \symbols
\XeTeXcharclass `^^^^2652 \symbols
\XeTeXcharclass `^^^^2653 \symbols
\XeTeXcharclass `^^^^271b \symbols
\XeTeXcharclass `^^^^2731 \symbols
\XeTeXcharclass `^^^^2733 \symbols
\XeTeXcharclass `^^^^27c0 \symbols
\XeTeXcharclass `^^^^27c1 \symbols
\XeTeXcharclass `^^^^27d8 \symbols
\XeTeXcharclass `^^^^27e6 \symbols
\XeTeXcharclass `^^^^27e7 \symbols
\XeTeXcharclass `^^^^27ea \symbols
\XeTeXcharclass `^^^^27eb \symbols
\XeTeXcharclass `^^^^2a5a \symbols
\XeTeXcharclass `^^^^2c80 \symbols
\XeTeXcharclass `^^^^2e00 \symbols
\XeTeXcharclass `^^^^2e01 \symbols
\XeTeXcharclass `^^^^2e02 \symbols
\XeTeXcharclass `^^^^2e03 \symbols
\XeTeXcharclass `^^^^2e04 \symbols
\XeTeXcharclass `^^^^2e05 \symbols
\XeTeXcharclass `^^^^2e06 \symbols
\XeTeXcharclass `^^^^2e07 \symbols
\XeTeXcharclass `^^^^2e09 \symbols
\XeTeXcharclass `^^^^2e0a \symbols
\XeTeXcharclass `^^^^2e0b \symbols
\XeTeXcharclass `^^^^2e0c \symbols
\XeTeXcharclass `^^^^2e0e \symbols
\XeTeXcharclass `^^^^2e0f \symbols
\XeTeXcharclass `^^^^2e10 \symbols
\XeTeXcharclass `^^^^2e11 \symbols
\XeTeXcharclass `^^^^2e12 \symbols
\XeTeXcharclass `^^^^2e13 \symbols
\XeTeXcharclass `^^^^2e14 \symbols
\XeTeXcharclass `^^^^2e15 \symbols
\XeTeXcharclass `^^^^2e16 \symbols
\XeTeXcharclass `^^^^2e20 \symbols
\XeTeXcharclass `^^^^2e20 \symbols
\XeTeXcharclass `^^^^2e21 \symbols
\XeTeXcharclass `^^^^2e21 \symbols
\XeTeXcharclass `^^^^2e26 \symbols
\XeTeXcharclass `^^^^2e27 \symbols
\XeTeXcharclass `^^^^2e28 \symbols
\XeTeXcharclass `^^^^2e29 \symbols
\XeTeXcharclass `^^^^2e2e \symbols
\XeTeXcharclass `^^^^^10110 \symbols
\XeTeXcharclass `^^^^^10111 \symbols
\XeTeXcharclass `^^^^^10112 \symbols
\XeTeXcharclass `^^^^^10112 \symbols
\XeTeXcharclass `^^^^^10140 \symbols
\XeTeXcharclass `^^^^^10141 \symbols
\XeTeXcharclass `^^^^^10142 \symbols
\XeTeXcharclass `^^^^^10143 \symbols
\XeTeXcharclass `^^^^^10144 \symbols
\XeTeXcharclass `^^^^^10145 \symbols
\XeTeXcharclass `^^^^^10146 \symbols
\XeTeXcharclass `^^^^^10147 \symbols
\XeTeXcharclass `^^^^^10148 \symbols
\XeTeXcharclass `^^^^^10149 \symbols
\XeTeXcharclass `^^^^^1014a \symbols
\XeTeXcharclass `^^^^^1014b \symbols
\XeTeXcharclass `^^^^^1014c \symbols
\XeTeXcharclass `^^^^^1014d \symbols
\XeTeXcharclass `^^^^^1014e \symbols
\XeTeXcharclass `^^^^^1014f \symbols
\XeTeXcharclass `^^^^^10150 \symbols
\XeTeXcharclass `^^^^^10151 \symbols
\XeTeXcharclass `^^^^^10152 \symbols
\XeTeXcharclass `^^^^^10153 \symbols
\XeTeXcharclass `^^^^^10154 \symbols
\XeTeXcharclass `^^^^^10155 \symbols
\XeTeXcharclass `^^^^^10156 \symbols
\XeTeXcharclass `^^^^^10157 \symbols
\XeTeXcharclass `^^^^^10158 \symbols
\XeTeXcharclass `^^^^^1015b \symbols
\XeTeXcharclass `^^^^^1015e \symbols
\XeTeXcharclass `^^^^^10166 \symbols
\XeTeXcharclass `^^^^^10175 \symbols
\XeTeXcharclass `^^^^^10176 \symbols
\XeTeXcharclass `^^^^^10177 \symbols
\XeTeXcharclass `^^^^^10179 \symbols
\XeTeXcharclass `^^^^^1017b \symbols
\XeTeXcharclass `^^^^^1017c \symbols
\XeTeXcharclass `^^^^^1017d \symbols
\XeTeXcharclass `^^^^^1017e \symbols
\XeTeXcharclass `^^^^^1017f \symbols
\XeTeXcharclass `^^^^^10180 \symbols
\XeTeXcharclass `^^^^^10183 \symbols
\XeTeXcharclass `^^^^^10184 \symbols
\XeTeXcharclass `^^^^^10185 \symbols
\XeTeXcharclass `^^^^^10186 \symbols
\XeTeXcharclass `^^^^^10188 \symbols
\XeTeXcharclass `^^^^^10189 \symbols
\XeTeXcharclass `^^^^^1018a \symbols
\XeTeXcharclass `^^^^^1d200 \symbols
\XeTeXcharclass `^^^^^1d201 \symbols
\XeTeXcharclass `^^^^^1d202 \symbols
\XeTeXcharclass `^^^^^1d203 \symbols
\XeTeXcharclass `^^^^^1d204 \symbols
\XeTeXcharclass `^^^^^1d205 \symbols
\XeTeXcharclass `^^^^^1d206 \symbols
\XeTeXcharclass `^^^^^1d207 \symbols
\XeTeXcharclass `^^^^^1d208 \symbols
\XeTeXcharclass `^^^^^1d209 \symbols
\XeTeXcharclass `^^^^^1d20a \symbols
\XeTeXcharclass `^^^^^1d20b \symbols
\XeTeXcharclass `^^^^^1d20c \symbols
\XeTeXcharclass `^^^^^1d20d \symbols
\XeTeXcharclass `^^^^^1d20e \symbols
\XeTeXcharclass `^^^^^1d20f \symbols
\XeTeXcharclass `^^^^^1d210 \symbols
\XeTeXcharclass `^^^^^1d211 \symbols
\XeTeXcharclass `^^^^^1d212 \symbols
\XeTeXcharclass `^^^^^1d213 \symbols
\XeTeXcharclass `^^^^^1d214 \symbols
\XeTeXcharclass `^^^^^1d215 \symbols
\XeTeXcharclass `^^^^^1d216 \symbols
\XeTeXcharclass `^^^^^1d217 \symbols
\XeTeXcharclass `^^^^^1d218 \symbols
\XeTeXcharclass `^^^^^1d219 \symbols
\XeTeXcharclass `^^^^^1d21a \symbols
\XeTeXcharclass `^^^^^1d21b \symbols
\XeTeXcharclass `^^^^^1d21c \symbols
\XeTeXcharclass `^^^^^1d21d \symbols
\XeTeXcharclass `^^^^^1d21e \symbols
\XeTeXcharclass `^^^^^1d21f \symbols
\XeTeXcharclass `^^^^^1d220 \symbols
\XeTeXcharclass `^^^^^1d221 \symbols
\XeTeXcharclass `^^^^^1d222 \symbols
\XeTeXcharclass `^^^^^1d223 \symbols
\XeTeXcharclass `^^^^^1d224 \symbols
\XeTeXcharclass `^^^^^1d225 \symbols
\XeTeXcharclass `^^^^^1d226 \symbols
\XeTeXcharclass `^^^^^1d227 \symbols
\XeTeXcharclass `^^^^^1d228 \symbols
\XeTeXcharclass `^^^^^1d229 \symbols
\XeTeXcharclass `^^^^^1d22a \symbols
\XeTeXcharclass `^^^^^1d22c \symbols
\XeTeXcharclass `^^^^^1d22d \symbols
\XeTeXcharclass `^^^^^1d22e \symbols
\XeTeXcharclass `^^^^^1d22f \symbols
\XeTeXcharclass `^^^^^1d230 \symbols
\XeTeXcharclass `^^^^^1d231 \symbols
\XeTeXcharclass `^^^^^1d232 \symbols
\XeTeXcharclass `^^^^^1d233 \symbols
\XeTeXcharclass `^^^^^1d234 \symbols
\XeTeXcharclass `^^^^^1d235 \symbols
\XeTeXcharclass `^^^^^1d236 \symbols
\XeTeXcharclass `^^^^^1d237 \symbols
\XeTeXcharclass `^^^^^1d238 \symbols
\XeTeXcharclass `^^^^^1d239 \symbols
\XeTeXcharclass `^^^^^1d23a \symbols
\XeTeXcharclass `^^^^^1d23d \symbols
\XeTeXcharclass `^^^^^1d23e \symbols
\XeTeXcharclass `^^^^^1d23f \symbols
\XeTeXcharclass `^^^^^1d240 \symbols
\XeTeXcharclass `^^^^^1d241 \symbols
\XeTeXcharclass `^^^^^1d242 \symbols
\XeTeXcharclass `^^^^^1d242 \symbols
\XeTeXcharclass `^^^^^1d243 \symbols
\XeTeXcharclass `^^^^^1d244 \symbols
\XeTeXcharclass `^^^^^1d245 \symbols
\XeTeXcharclass `^^^^^1d510 \symbols
\XeTeXcharclass `^^^^^1d516 \symbols
\XeTeXcharclass `^^^^^f022e \symbols

%------------------------------------------------------------------------
% 0 = Latin or Greek text
% \symbols is the class of chars defined above
% 4095 = text boudary, e.g. space 
% In older Xetex versions this was 255
% see comments in 
% /usr/local/texlive/2016/texmf-dist/tex/generic/tex-ini-files/xelatex.ini
%------------------------------------------------------------------------

% transitions defined for when a member of \symbols
% is encountered
% \symbol is the font (usualy Cardo) defined in the tex file

\XeTeXinterchartoks 0 \symbols {\Symbol }
\XeTeXinterchartoks \symbols 0 {\rmfamily }

\XeTeXinterchartoks 4095 \symbols {\Symbol }
\XeTeXinterchartoks \symbols 4095 {\rmfamily }




\begin{document}
\begin{center}
\greek
\fbox{\fbox{\fbox{ {\large \textbf{Τεκμήριον λειτουργίας τοῦ} \XeLaTeX}}}}
\\
\vspace*{5mm}
\latin
{\footnotesize This is a test of the correct functioning of the \XeLaTeX\ installation.}
\\
{\tiny The text is taken from Plutarch᾽s De Esu Carnium I}
\end{center}
\greek
\xentrystretch{0}
\begin{center}
\begin{xtabular}{p{1cm}p{10.5454545454545cm}p{0.8cm}p{0cm}}
&\textbf{ΠΕΡΙ ΣΑΡΚΟΦΑΓΙΑΣ} &&\\
&\textbf{ΛΟΓΟΣ Αʹ} &&\\
&Ἀλλὰ σὺ μὲν ἐρωτᾷς τίνι λόγῳ Πυθαγόρας ἀπείχετο &&\\
&σαρκοφαγίας, ἐγὼ δὲ θαυμάζω καὶ τίνι πάθει καὶ ποίᾳ &&\\
&ψυχῇ [ἢ λόγῳ] ὁ πρῶτος ἄνθρωπος ἥψατο φόνου στόματι &&\\
&καὶ τεθνηκότος ζῴου χείλεσι προσήψατο σαρκὸς καὶ νεκ-&&\\
&ρῶν σωμάτων καὶ ἑώλων προθέμενος τραπέζας ὄψα καὶ &&\\
&τρυφὰς [καὶ] προσέτι εἶπεν τὰ μικρὸν ἔμπροσθεν βρυχώ-&&\\
&μενα μέρη καὶ φθεγγόμενα καὶ κινούμενα καὶ βλέποντα·  &&\\
&πῶς ἡ ὄψις ὑπέμεινε τὸν φόνον σφαζομένων δερομένων &&\\
&διαμελιζομένων, πῶς ἡ ὄσφρησις ἤνεγκε τὴν ἀποφοράν, &&\\
&πῶς τὴν γεῦσιν οὐκ ἀπέστρεψεν ὁ μολυσμὸς ἑλκῶν ψαύου-&&\\
&σαν ἀλλοτρίων καὶ τραυμάτων θανασίμων χυμοὺς καὶ &&\\
&ἰχῶρας ἀπολαμβάνουσαν. (μ  \latin{}395 \greek{}) &&\\
&
 &&\\
&‘εἷρπον μὲν ῥινοί, κρέα δ’ ἀμφ’ ὀβελοῖς ἐμεμύκει &&\\
&ὀπταλέα τε καὶ ὠμά, βοῶν δ’ ὣς γίγνετο φωνή·’ &&\\
&
 &&\\
&τοῦτο πλάσμα καὶ μῦθός ἐστι, τὸ δέ γε δεῖπνον ἀληθῶς &&\\
&τερατῶδες, πεινῆν τινα τῶν μυκωμένων ἔτι, [καὶ] διδά-&&\\
&σκοντα ἀφ’ ὧν δεῖ τρέφεσθαι ζώντων ἔτι καὶ λαλούντων, &&\\
&〈καὶ〉 διαταττόμενον ἀρτύσεις τινὰς καὶ ὀπτήσεις καὶ &&\\
&παραθέσεις· τούτων ἔδει ζητεῖν τὸν πρῶτον ἀρξάμενον &&\\
&οὐ τὸν ὀψὲ παυσάμενον. &&\\
&Ἢ τοῖς μὲν πρώτοις ἐκείνοις ἐπιχειρήσασι σαρκο-&&\\
&φαγεῖν τὴν αἰτίαν ἂν εἴποι τις εἶναι τὴν ἀπορίαν· οὐ γὰρ &&\\
&ἐπιθυμίαις ἀνόμοις συνδιάγοντες οὐδ’ ἐν περιουσίᾳ τῶν &&\\
&ἀναγκαίων ὑβρίσαντες εἰς ἡδονὰς παρὰ φύσιν ἀσυμφύ-&&\\
&λους ἐπὶ ταῦτ’ ἦλθον· ἀλλ’ εἴποιεν ἂν αἴσθησιν ἐν τῷ &&\\
&παρόντι καὶ φωνὴν λαβόντες· ‘ὦ μακάριοι καὶ θεοφιλεῖς &&\\
&οἱ νῦν ὄντες ὑμεῖς, οἷον βίου λαχόντες αἰῶνα καρποῦσθε &&\\
&καὶ νέμεσθε κλῆρον ἀγαθῶν ἄφθονον· ὅσα φύεται ὑμῖν,  &&\\
&ὅσα τρυγᾶται· ὅσον πλοῦτον ἐκ πεδίων, ὅσας ἀπὸ φυτῶν &&\\
&ἡδονὰς [ἃς] δρέπεσθαι πάρεστιν· ἔξεστιν ὑμῖν καὶ τρυφᾶν &&\\
&μὴ μιαινομένοις. ἡμᾶς δὲ σκυθρωπότατον καὶ φοβερώτα-&&\\
&τον ἐδέξατο βίου καὶ χρόνου μέρος, εἰς πολλὴν καὶ ἀμήχα-&&\\
&νον ἐκπεσόντας ἀπὸ τῆς πρώτης γενέσεως ἀπορίαν· ἔτι &&\\
&μὲν οὐρανὸν ἔκρυπτεν ἀὴρ καὶ ἄστρα, θολερῷ καὶ δυσδια-&&\\
&στατοῦντι πεφυρμένος ὑγρῷ καὶ πυρὶ καὶ ζάλαις ἀνέμων· &&\\
&’οὔπω δ’ ἥλιος’ ἵδρυτο ἀπλανὴς καὶ βέβαιον &&\\
&
 &&\\
&‘ἔχων δρόμον, ἠῶ &&\\
&καὶ δύσιν ἔκρινεν, περὶ δ’ ἤγαγεν αὖθις ὀπίσσω &&\\
&καρποφόροισιν ἐπιστέψας καλυκοστεφάνοισιν &&\\
&Ὥραις, γῆ δ’ ὕβριστο’ ( \latin{}Empedocl. B 154 \greek{}) &&\\
&
 &&\\
&ποταμῶν ἐκβολαῖς ἀτάκτοις, καὶ πολλὰ ‘λίμναισιν ἄμορ-&&\\
&φα’ καὶ πηλοῖς βαθέσι καὶ λόχμαις ἀφόροις καὶ ὕλαις ἐξη-&&\\
&γρίωτο· φορὰ δ’ ἡμέρων καρπῶν καὶ τέχνης ὄργανον &&\\
&οὐδὲν 〈ἦν〉 οὐδὲ μηχανὴ σοφίας· ὁ δὲ λιμὸς οὐκ ἐδίδου &&\\
&χρόνον οὐδ’ ὥρας ἐτησίους σπόρος † ὢν τότ’ ἀνέμενε. τί &&\\
&θαυμαστόν, εἰ ζῴων ἐχρησάμεθα σαρξὶ παρὰ φύσιν, ὅτ’ &&\\
&ἰλὺς ἠσθίετο ‘καὶ φλοιὸς ἐβρώθη ξύλου’, καὶ ‘ἄγρωστιν &&\\
&εὑρεῖν βλαστάνουσαν ἢ φλεώ’ τινα ῥίζαν εὐτυχὲς ἦν; βαλά-&&\\
&νου δὲ γευσάμενοι καὶ φαγόντες ἐχορεύσαμεν ὑφ’ ἡδονῆς &&\\
&περὶ δρῦν τινα καὶ φηγόν, ζείδωρον καὶ μητέρα καὶ τρο- &&\\
&φὸν ἀποκαλοῦντες· ἐκείνην | [ἣν] ὁ τότε βίος ἑορτὴν ἔγνω, &&\\
&τὰ δ’ ἄλλα φλεγμονῆς ἦν ἅπαντα μεστὰ καὶ στυγνότητος. &&\\
&ὑμᾶς δὲ τοὺς νῦν τίς λύσσα καὶ τίς οἶστρος ἄγει πρὸς μιαι-&&\\
&φονίαν, οἷς τοσαῦτα περίεστι τῶν ἀναγκαίων; τί κατα-&&\\
&ψεύδεσθε τῆς γῆς ὡς τρέφειν μὴ δυναμένης; τί τὴν θεσμο-&&\\
&φόρον ἀσεβεῖτε Δήμητραν καὶ τὸν ἡμερίδην καὶ μειλίχιον &&\\
&αἰσχύνετε Διόνυσον, ὡς οὐχ ἱκανὰ παρὰ τούτων λαμβά-&&\\
&νοντες; οὐκ αἰδεῖσθε τοὺς ἡμέρους καρποὺς αἵματι καὶ &&\\
&φόνῳ μιγνύοντες; ἀλλὰ δράκοντας ἀγρίους καλεῖτε καὶ &&\\
&παρδάλεις καὶ λέοντας, αὐτοὶ δὲ μιαιφονεῖτ’ εἰς ὠμότητα &&\\
&καταλιπόντες ἐκείνοις οὐδέν· ἐκείνοις μὲν γὰρ ὁ φόνος &&\\
&τροφή, ὑμῖν δ’ ὄψον ἐστίν‘. &&\\
&*** Οὐ γὰρ δὴ λέοντάς γ’ ἀμυνόμενοι καὶ λύκους &&\\
&ἐσθίομεν· ἀλλὰ ταῦτα μὲν ἐῶμεν, τὰ δ’ ἀβλαβῆ καὶ χει-&&\\
&ροήθη καὶ ἄκεντρα καὶ νωδὰ πρὸς τὸ δακεῖν συλλαμβά-&&\\
&νοντες ἀποκτιννύομεν, ἃ νὴ Δία καὶ κάλλους ἕνεκα καὶ &&\\
&χάριτος ἡ φύσις ἔοικεν ἐξενεγκεῖν. *** &&\\
&Ὅμοιον ὡς εἴ τις τὸν Νεῖλον ὁρῶν πλημμυροῦντα καὶ &&\\
&τὴν χώραν ἐμπιπλάντα γονίμου καὶ καρποφόρου ῥεύματος &&\\
&μὴ τοῦτο θαυμάζοι τοῦ φερομένου, τὸ φυτάλμιον καὶ &&\\
&εὔφορον τῶν ἡμερωτάτων καὶ βιωφελεστάτων καρπῶν,  &&\\
&ἀλλ’ ἰδών που καὶ κροκόδειλον ἐννηχόμενον καὶ ἀσπίδα &&\\
&κατασυρομένην καὶ μύας, ἄγρια ζῷα καὶ μιαρά, ταύτας &&\\
&λέγοι τὰς αἰτίας τῆς μέμψεως κατὰ τῆς τοῦ πράγματος &&\\
&ἀνάγκης· ἢ νὴ Δία τὴν γῆν ταύτην καὶ τὴν ἄρουραν ἀπο-&&\\
&βλέψας ἐμπεπλησμένην ἡμέρων καρπῶν καὶ βρίθουσαν &&\\
&ἀσταχύων, ἔπειθ’ ὑποβλέψας που τοῖς ληίοις τούτοις καί &&\\
&πού τινος αἴρας στάχυν 〈ἐν〉ιδὼν καὶ ὀροβάγχην, εἶτ’ &&\\
&ἀφεὶς ἐκεῖνα καρποῦσθαι καὶ ληίζεσθαι μέμφοιτο περὶ &&\\
&τούτων. τί οὖν, εἰ καὶ λόγον ῥήτορος ὁρῶν ἐν δίκῃ τινὶ &&\\
&καὶ συνηγορίᾳ πληθύοντα καὶ φερόμενον ἐπὶ βοηθείᾳ κιν-&&\\
&δύνων ἢ νὴ Δί’ ἐλέγχῳ καὶ κατηγορίᾳ τολμημάτων καὶ &&\\
&† ἀποδείξεων, ῥέοντα δὲ καὶ φερόμενον οὐχ ἁπλῶς οὐδὲ &&\\
&λιτῶς, ἀλλ’ ὁμοῦ πάθεσι πολλοῖς μᾶλλον δὲ παντοδαποῖς, &&\\
&εἰς ψυχὰς ὁμοίως πολλὰς καὶ ποικίλας καὶ διαφόρους τῶν &&\\
&ἀκροωμένων ἢ τῶν δικαζόντων, ἃς δεῖ τρέψαι καὶ μετα-&&\\
&βαλεῖν ἢ νὴ Δία πραῧναι καὶ ἡμερῶσαι καὶ καταστῆσαι, &&\\
&εἶτα παρεὶς τοῦτο τοῦ πράγματος ὁρᾶν καὶ μετρεῖν τὸ &&\\
&† φύλαιον καὶ ἀγώνισμα, παραρρήσεις ἐκλέγοι, ἃς κατιὼν &&\\
&ὁ λόγος συγκατήνεγκε τῇ ῥύμῃ τῆς φορᾶς, συνεκπεσούσας  &&\\
&καὶ παρολισθούσας τῷ λοιπῷ τοῦ λόγου; καὶ δημηγόρου &&\\
&τινὸς ὁρῶν *** &&\\
&Ἀλλ’ οὐδὲν ἡμᾶς δυσωπεῖ, οὐ χρόας ἀνθηρὸν εἶδος, &&\\
&οὐ φωνῆς ἐμμελοῦς πιθανότης, οὐ πανουργία ψυχῆς, οὐ &&\\
&τὸ καθάριον ἐν διαίτῃ καὶ περιττὸν ἐν συνέσει τῶν ἀθλίων, &&\\
&ἀλλὰ σαρκιδίου μικροῦ χάριν ἀφαιρούμεθα ψυχήν, ἡλίου &&\\
&φῶς, τὸν τοῦ βίου χρόνον, ἐφ’ ἃ γέγονε καὶ πέφυκεν. εἶθ’ &&\\
&ἃς φθέγγεται καὶ διατρίζει φωνὰς ἀνάρθρους εἶναι δοκοῦ-&&\\
&μεν, οὐ παραιτήσεις καὶ δεήσεις καὶ δικαιολογίας ἑκάστου &&\\
&λέγοντος ’οὐ παραιτοῦμαί σου τὴν ἀνάγκην ἀλλὰ τὴν &&\\
&ὕβριν· ἵνα φάγῃς ἀπόκτεινον, ἵνα δ’ ἥδιον φάγῃς μή μ’ &&\\
&ἀναίρει.’ ὢ τῆς ὠμότητος· δεινὸν μέν ἐστι καὶ τιθεμένην &&\\
&ἰδεῖν τράπεζαν ἀνθρώπων πλουσίων νεκροκόσμοις χρω-&&\\
&μένων μαγείροις καὶ ὀψοποιοῖς, δεινότερον δ’ ἀποκομιζο-&&\\
&μένην· πλείονα γὰρ τὰ λειπόμενα τῶν βεβρωμένων ἐστίν· &&\\
&οὐκοῦν ταῦτα μάτην ἀπέθανεν· ἕτερα δὲ φειδόμενοι τῶν &&\\
&παρατεθέντων οὐκ ἐῶσι τέμνειν οὐδὲ κατακόπτειν, παραι-&&\\
&τούμενοι νεκρά, ζώντων δ’ οὐκ ἐφείσαντο. &&\\
&Ἀλλ’ ἄγε παρειλήφαμεν ἐκείνους λέγειν τοὺς ἄνδρας &&\\
&ἀρχὴν ἔχειν τὴν φύσιν *** ὅτι γὰρ οὐκ ἔστιν ἀνθρώπῳ &&\\
&κατὰ φύσιν τὸ σαρκοφαγεῖν, πρῶτον μὲν ἀπὸ τῶν σωμά-&&\\
&των δηλοῦται τῆς κατασκευῆς. οὐδενὶ γὰρ ἔοικε τὸ ἀν-&&\\
&θρώπου σῶμα τῶν ἐπὶ σαρκοφαγίᾳ γεγονότων, οὐ γρυ-&&\\
&πότης χείλους, οὐκ ὀξύτης ὄνυχος, | οὐ τραχύτης ὀδόντος  &&\\
&πρόσεστιν, οὐ κοιλίας εὐτονία καὶ πνεύματος θερμότης &&\\
&τρέψαι καὶ κατεργάσασθαι δυνατὴ τὸ βαρὺ καὶ κρεῶδες· &&\\
&ἀλλ’ αὐτόθεν ἡ φύσις τῇ λειότητι τῶν ὀδόντων καὶ τῇ &&\\
&σμικρότητι τοῦ στόματος καὶ τῇ μαλακότητι τῆς γλώσ-&&\\
&σης καὶ τῇ πρὸς πέψιν ἀμβλύτητι τοῦ πνεύματος ἐξόμνυ-&&\\
&ται τὴν σαρκοφαγίαν. εἰ δὲ λέγεις πεφυκέναι σεαυτὸν ἐπὶ &&\\
&τοιαύτην ἐδωδήν, ὃ βούλει φαγεῖν πρῶτον αὐτὸς ἀπόκτει-&&\\
&νον, ἀλλ’ αὐτὸς διὰ σεαυτοῦ, μὴ χρησάμενος κοπίδι μηδὲ &&\\
&τυμπάνῳ τινὶ μηδὲ πελέκει· ἀλλ’, ὡς λύκοι καὶ ἄρκτοι &&\\
&καὶ λέοντες αὐτοί, ὅσα ἐσθίουσι, φονεύουσιν, ἄνελε δή-&&\\
&γματι βοῦν ἢ στόματι σῦν, ἢ ἄρν’ ἢ λαγωὸν διάρρηξον &&\\
&καὶ φάγε προσπεσὼν ἔτι ζῶντος, ὡς ἐκεῖνα. εἰ δ’ ἀνα-&&\\
&μένεις νεκρὸν γενέσθαι τὸ ἐσθιόμενον καὶ δυσωπεῖ σε &&\\
&παροῦσα ψυχὴ ἀπολαύειν τῆς σαρκός, τί παρὰ φύσιν ἐσθίεις &&\\
&τὸ ἄψυχον; ἀλλ’ οὐδ’ ἄψυχον ἄν τις φάγοι καὶ νεκρὸν οἷόν &&\\
&ἐστιν, ἀλλ’ ἕψουσιν ὀπτῶσι μεταβάλλουσι διὰ πυρὸς καὶ &&\\
&φαρμάκων, ἀλλοιοῦντες καὶ τρέποντες καὶ σβεννύοντες &&\\
&ἡδύσμασι μυρίοις τὸν φόνον, ἵν’ ἡ γεῦσις ἐξαπατηθεῖσα &&\\
&προσδέξηται τὸ ἀλλότριον. καίτοι χάριέν γε τὸ τοῦ &&\\
&Λάκωνος, ὃς ἰχθύδιον ἐν πανδοκείῳ πριάμενος τῷ παν-&&\\
&δοκεῖ σκευάσαι παρέδωκεν· αἰτοῦντος δ’ ἐκείνου τυρὸν &&\\
&καὶ ὄξος καὶ ἔλαιον, ‘ἀλλ’ εἰ ταῦτ’ εἶχον’ εἶπεν ‘οὐκ ἂν &&\\
&ἰχθὺν ἐπριάμην’. ἡμεῖς δ’ οὕτως ἐν τῷ μιαιφόνῳ τρυφῶ-&&\\
&μεν, ὥστ’ ὄψον τὸ κρέας προσαγορεύομεν, εἶτ’ ὄψων πρὸς &&\\
&αὐτὸ τὸ κρέας δεόμεθα, ἀναμιγνύντες ἔλαιον οἶνον μέλι &&\\
&γάρον ὄξος ἡδύσμασι Συριακοῖς Ἀραβικοῖς, ὥσπερ ὄντως  &&\\
&νεκρὸν ἐνταφιάζοντες. καὶ γὰρ οὕτως αὐτῶν διαλυθέντων &&\\
&καὶ μαλαχθέντων καὶ τρόπον τινὰ προσαπέντων ἔργον &&\\
&ἐστὶ τὴν πέψιν κρατῆσαι, καὶ διακρατησάσης δὲ δεινὰς &&\\
&βαρύτητας ἐμποιεῖ καὶ νοσώδεις ἀπεψίας. Διο-&&\\
&γένης δ’ ὠμὸν φαγεῖν πολύπουν ἐτόλμησεν, ἵνα τὴν διὰ &&\\
&τοῦ πυρὸς ἐκβάλῃ κατεργασίαν τῶν κρεῶν· καὶ πολλῶν &&\\
&περιεστώτων αὐτὸν ἀνθρώπων, ἐγκαλυψάμενος τῷ τρί-&&\\
&βωνι καὶ τῷ στόματι προσφέρων τὸ κρέας ‘ὑπὲρ ὑμῶν’ &&\\
&φησίν ‘ἐγὼ παραβάλλομαι καὶ προκινδυνεύω.’ καλόν, ὦ &&\\
&Ζεῦ, κίνδυνον· οὐ γάρ, ὡς Πελοπίδας ὑπὲρ τῆς Θηβῶν &&\\
&ἐλευθερίας ἢ ὡς Ἁρμόδιος καὶ Ἀριστογείτων ὑπὲρ Ἀθη-&&\\
&ναίων, προεκινδύνευσεν ὁ φιλόσοφος ὠμῷ πολύποδι δια-&&\\
&μαχόμενος, ἵνα τὸν βίον ἀποθηριώσῃ; οὐ τοίνυν &&\\
&μόνον αἱ κρεοφαγίαι τοῖς σώμασι γίγνονται παρὰ φύσιν, &&\\
&ἀλλὰ καὶ τὰς ψυχὰς ὑπὸ πλησμονῆς καὶ κόρου παχύνου-&&\\
&σιν· ‘οἶνος γὰρ καὶ σαρκῶν ἐμφορήσιες σῶμα μὲν ἰσχυ-&&\\
&ρὸν ποιέουσι καὶ ῥωμαλέον, ψυχὴν δὲ ἀσθενέα.’ καὶ ἵνα &&\\
&μὴ τοῖς ἀθληταῖς ἀπεχθάνωμαι, συγγενέσι χρῶμαι παρα-&&\\
&δείγμασι· τοὺς γὰρ Βοιωτοὺς ἡμᾶς οἱ Ἀττικοὶ καὶ παχεῖς &&\\
&καὶ ἀναισθήτους καὶ ἠλιθίους μάλιστα διὰ τὰς ἀδηφαγίας &&\\
&προσηγόρευον· ‘οὗτοι δ’ αὖ † σῦς *** καὶ ὁ Μένανδρος &&\\
&( \latin{}fr. 748 Koerte \greek{}) † οἳ γνάθους ἔχουσι,’ καὶ ὁ Πίνδαρος  &&\\
&( \latin{}Ol. VI 89 sq. \greek{}) ‘γνῶναί τ’ ἔπειτα ***’ ‘αὐγὴ ξηρὴ ψυχὴ &&\\
&σοφωτάτη’ κατὰ τὸν Ἡράκλειτον ( \latin{}B 118 \greek{})· οἱ κενοὶ πίθοι &&\\
&κρουσθέντες ἠχοῦσι, γενόμενοι δὲ πλήρεις οὐχ ὑπακού-&&\\
&ουσι ταῖς πληγαῖς· τῶν χαλκωμάτων τὰ λεπτὰ τοὺς ψό-&&\\
&φους ἐν κύκλῳ διαδίδωσιν, ἄχρις οὗ ἐμφράξῃ καὶ τυφλώσῃ &&\\
&〈τις〉 τῇ χειρὶ τῆς πληγῆς περιφερομένης ἐπιλαμβανό-&&\\
&μενος· ὀφθαλμὸς ὑγροῦ πλεονάσαντος ἀναπλησθεὶς μα-&&\\
&ραυγεῖ καὶ ἀτονεῖ πρὸς τὸ οἰκεῖον ἔργον· τὸν ἥλιον δι’ &&\\
&ἀέρος ὑγροῦ καὶ πλῆθος ἀναθυμιάσεων ἀπέπτων ἀθροί-&&\\
&σαντος οὐ καθαρὸν οὐδὲ λαμπρὸν ἀλλὰ βύθιον καὶ ἀχλυ-&&\\
&ώδη καὶ ὀλισθάνοντα ταῖς αὐγαῖς ὁρῶμεν. οὕτω δὴ καὶ &&\\
&διὰ σώματος θολεροῦ καὶ διακόρου καὶ βαρυνομένου &&\\
&τροφαῖς ἀσυμφύλοις | πᾶς’ ἀνάγκη τὸ γάνωμα τῆς &&\\
&ψυχῆς καὶ τὸ φέγγος ἀμβλύτητα καὶ σύγχυσιν ἔχειν &&\\
&καὶ πλανᾶσθαι καὶ φέρεσθαι, πρὸς τὰ λεπτὰ καὶ &&\\
&δυσθεώρητα τέλη τῶν πραγμάτων αὐγὴν καὶ τόνον οὐκ &&\\
&ἐχούσης. &&\\
&Χωρὶς δὲ τούτων ὁ πρὸς φιλανθρωπίαν ἐθισμὸς οὐ &&\\
&δοκεῖ θαυμαστὸν εἶναι; τίς γὰρ ἂν ἀδικήσειεν ἄνθρωπον &&\\
&οὕτω πρὸς ἀλλότρια καὶ ἀσύμφυλα διακείμενος [καὶ]  &&\\
&πράως καὶ φιλανθρώπως; ἐμνήσθην δὲ τρίτην ἡμέραν &&\\
&διαλεγόμενος τὸ τοῦ Ξενοκράτους ( \latin{}fr. 99 H. \greek{}), καὶ ὅτι &&\\
&Ἀθηναῖοι τῷ ζῶντα τὸν κριὸν ἐκδείραντι δίκην ἐπέθηκαν· &&\\
&οὐκ ἔστι δ’, οἶμαι, χείρων ὁ ζῶντα βασανίζων τοῦ παραι-&&\\
&ρουμένου τὸ ζῆν καὶ φονεύοντος, ἀλλὰ μᾶλλον, ὡς ἔοικε, &&\\
&τῶν παρὰ συνήθειαν ἢ τῶν παρὰ φύσιν αἰσθανόμεθα. &&\\
&καὶ ταῦτα μὲν ἐκεῖ κοινότερον ἔλεγον· τὴν δὲ μεγάλην καὶ &&\\
&μυστηριώδη καὶ ἄπιστον ἀνδράσι δεινοῖς, ᾗ φησιν ὁ &&\\
&Πλάτων ( \latin{}Phaedr. 245c \greek{}), καὶ θνητὰ φρονοῦσιν ἀρχὴν &&\\
&τοῦ δόγματος ὀκνῶ μὲν ἔτι τῷ λόγῳ κινεῖν, ὥσπερ ναῦν &&\\
&ἐν χειμῶνι ναύκληρος ἢ μηχανὴν αἴρειν ποιητικὸς ἀνὴρ &&\\
&ἐν θεάτρῳ σκηνῆς περιφερομένης. οὐ χεῖρον δ’ ἴσως καὶ &&\\
&προανακρούσασθαι καὶ προαναφωνῆσαι τὰ τοῦ Ἐμπεδο-&&\\
&κλέους· *** ἀλληγορεῖ γὰρ ἐνταῦθα τὰς ψυχάς, ὅτι &&\\
&φόνων καὶ βρώσεως σαρκῶν καὶ ἀλληλοφαγίας δίκην &&\\
&τίνουσαι σώμασι θνητοῖς ἐνδέδενται. καίτοι δοκεῖ παλαι-&&\\
&ότερος οὗτος ὁ λόγος εἶναι· τὰ γὰρ δὴ περὶ τὸν Διόνυσον &&\\
&μεμυθευμένα πάθη τοῦ διαμελισμοῦ καὶ τὰ Τιτάνων ἐπ’ &&\\
&αὐτὸν τολμήματα γευσαμένων τε τοῦ φόνου κολάσεις [τε  &&\\
&τούτων] καὶ κεραυνώσεις, ᾐνιγμένος ἐστὶ μῦθος εἰς τὴν &&\\
&παλιγγενεσίαν· τὸ γὰρ ἐν ἡμῖν ἄλογον καὶ ἄτακτον καὶ &&\\
&βίαιον οὐ θεῖον ἀλλὰ δαιμονικὸν οἱ παλαιοὶ Τιτᾶνας &&\\
&ὠνόμασαν, [καὶ] τοῦτ’ ἔστι κολαζομένους καὶ δίκην &&\\
&τίνοντας.  &&\\

  & & & \\
  &\Symbol\grtoday& & \\
  \end{xtabular}
  \end{center}
  \end{document}

